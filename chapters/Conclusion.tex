\section{Summary}
We propose a simple approach for visually salient region detection effectively where we use color differences between regions as well as spatial distance. Then for making the result more close to ground truth we introduce color importance.
After comparing color difference along with spatial distance between regions visually salient regions are highlighted.Then when comparing each region with boundary regions relatively less visually salient regions which are meant to be background are suppressed. Then when incorporating color importance , we find the mean color of the image and subtract that mean color from every region. We know backgrounds of an image tends to cover more area than foreground. So the mean color will be close to background region and subtraction will eventually suppress the background. We then combine global saliency map, boundary aware contrast map and color importance map to produce our final saliecy map. For generating superpixels we used SLIC which is a fast and convenient algorithm to generate superpixels effectively. We used The  MSRA Salient Object Database \cite{cheng2015global}\cite{SalObjSurvey}\cite{SalObjBenchmark}\cite{13iccv/Cheng_Saliency} to build output and generated precision recall curve to compare the output with ground truth statistically rather than doing it with visual perception.
\section{Limitation}
Some challenges are still needed to be taken into account. For example if part of an object has the same color as background then creating global contrast map,boundary aware contrast map and color importance map  won't be enough to detect that portion. Eventually we will end up with foreground object where the part of it which matched the background will be considered as background. This method is fast, simple yet effective approach to find visually salient regions from images
having simple background. however images having more complex background will fail to produce impressive result using this method. If the border has a frame with same color as foreground object then boundary aware contrast map doesn't contribute at all to detect visually salient region instead it push the output to a negative extent. For images with relatively simple background and foreground this method works fine.



\section{Future Scope}
In the future, we will study more on complex data set, as the data set eliminating center bias. This will make huge challenge to the existing saliency detection methods, may even open up a new research direction of saliency detection.
We will also try to eliminate  our limitations. As this is a preprocessing task of many computer vision related job, the more accurate it gets the result of those computer vision related jobs will also become more accurate.
\noindent
An important need of a robot is to know where it is located. For this aim, the robot can use the data from its sensors to find landmarks (salient features extraction) and register images taken at different times (salient features comparison) to build a model of the scene. The general process of real-time building of a view of the scene is called Simultaneous Localization and Mapping (SLAM). Saliency models can help a lot in the extraction of more stable landmarks from images which can be more robustly compared \cite{frintrop2008attentional}. Those techniques imply first the computation of saliency maps, but the results are not used directly: they need to be further processed (especially comparisons of salient areas).
Another important need of robots after they establish the scene, is to recognize the objects which are present in this scene and which might be interesting to interact with. Two steps are needed to recognize objects. First of all, the robot needs to detect the object in a scene. For this goal saliency models can help a lot as they can provide information about proto-objects or areas objectness . When objects are detected, they need to be recognized.
Among the different applications of automatic saliency computation, the marketing and communication optimization is probably one of the closest to market. As it is possible to predict an image attention map, which is a map of the probability that people attend each pixel of the image, it is possible to predict where people are likely to look on a marketing material like an advertisement or a website. Attracting customer attention is the first step of the process of attracting people interest, induce desire and need for the product and finally push the client to buy it.